%% LyX 2.4.3 created this file.  For more info, see https://www.lyx.org/.
%% Do not edit unless you really know what you are doing.
\documentclass[english]{article}
\usepackage[T1]{fontenc}
\usepackage[latin9]{inputenc}
\usepackage{babel}
\begin{document}

\section{Conclusion and Outlook}

The optimization of the fuel cell shows that a desired power density
curve can be approximated very well using genetic algorithms. Interestingly,
GA reveals a non-obvious structure of fuel cell performance, where
different parameter settings lead to the same power density curve.
This means that the solution is not unique here, which in turn could
be used for a specific desired parameter setting. For further investigations,
it would be important to include or couple the temperature generated
by the operation in the simulations, which for the sake of simplicity
was chosen as constant for this work. Knowledge of the effect of this
temperature on performance is desirable.

Next was the achievement of a statistically uniform and isotropic
field studied in an electromagnetic reverberation cuboid chamber,
with a Vivaldi antenna as source. The field turbulence was caused
by a stirrer whose geometry plays a crucial role in the field distribution.
In this work, the shape of the stirrer blades was optimized in order
to achieve the best possible desired field distribution. This enabled
the desired values to be achieved except for the real part of the
field in the y-direction. In order to further improve the field distribution
values, the number of stirrers was increased to three pieces. The
wing shapes have been optimized. The field in the y-direction could
be improved, but the field values in the other spatial directions
deteriorated but remained within the tolerance range. Nevertheless,
the desired acceptance limits for the field distribution in all spatial
directions were maintained. For further investigations it would also
be interesting to optimize the topology of the stirrer. To do this,
one does not assume a rectangular sheet metal as the basic shape of
a stirrer, but calculates the optimal shape completely mathematically.
In a further step, a different approach to improve the field distribution
was chosen. There for a certain number of two different geometric
objects (cone and sphere) were attached to the wall chamber and their
influence on the field distribution was examined. Here the hemispheres
provided a better field distribution than cones, so this would be
a better choice for further investigation. Here, the hemispheres provided
a better field distribution than the cones, so the hemisphere was
used for further investigations. Now an attempt was made to vary the
number of hemispheres randomly on the wall in the chamber, but unfortunately
this did not lead to a better distribution of the field. For further
investigation, one should perhaps not make a random change in the
number of geometric objects, but a systematic change. The influence
of multiple hemispheres with different randomly chosen radii in chamber
would be an interesting research project.
\end{document}
